\documentclass{beamer}

\usepackage{graphicx}

\usepackage{epstopdf}

% \usetheme{Rochester}
\usecolortheme{focus}
% % \usefonttheme{structuresmallcapsserif}
% \usefonttheme{focus}

\title{$P$-Wave Charmonia Production in Exclusive $B_c$-Meson Decays}
\author{Alexey Luchinsky}
\institute{IHEP, Russia\\ BGSU, Bowling Green, OH, USA}


\newcommand{\R}{\mathcal{R}}
\newcommand{\M}{\mathcal{M}}
\newcommand{\A}{\mathcal{A}}
	
\newcommand{\cc}{(\bar{c}c)}
	
\graphicspath{{figs/}}

\begin{document}

\begin{frame}
  \maketitle
\end{frame}

\section{Introduction}
\begin{frame}
  \frametitle{Introduction}
  \begin{itemize}
  \item $B_c^+ = (\bar{b}c)$ combines both charmonia and bottomonia properties
  \item Can be used to
    \begin{itemize}
    \item study QCD in confinment and assymtotoc freedom regimes
    \item check existing models for heavy quarkonia description      
    \end{itemize}
  \item $B_c \to \cc+\R$ decays can be studied using the factorization model
  \item Good results for vector charmonia:
    \begin{itemize}
    \item $B_c \to \psi^{(')} + e\nu, \pi, \rho, 3\pi, \dots$
    \end{itemize}
  \item What about $P$-wave states
    \begin{itemize}
    \item $B_c \to \chi_{cJ} + e\nu, \pi, \rho, 3\pi, \dots$
    \end{itemize}

  \end{itemize}
\end{frame}

\begin{frame}
  \frametitle{Content}
  \tableofcontents
\end{frame}


\begin{frame}[t]
  \frametitle{Factorization}
  Diagram and amplitude factorise:
  \begin{center}
    \includegraphics[width=0.4\columnwidth]{diags_BcCCW}
  \end{center}
      $$\M\left(B_c \to \cc + \R\right) = H^\mu \epsilon^{(\R)}_\mu$$
   where
   \begin{itemize}
   \item $H^\mu$ is the $B_c\to \cc W$ transition vertex
     \begin{itemize}
     \item Potential model, QCD sum rules, etc
     \end{itemize}
   \item $\epsilon^{(\R)}_\mu$ is effective  $W\to\R$ polarization vector
     \begin{itemize}
     \item ChPt, phonomenologr, resonance approximation, etc
     \end{itemize}
   \end{itemize}
The differential width is
$$
\frac{d\Gamma}{dq^2} \sim \frac{d\Gamma_T}{dq^2} \rho_T\left(q^2\right) + \frac{d\Gamma_L}{dq^2} \rho_L\left(q^2\right)
$$
\end{frame}

\section{$B_c\to \chi_{cJ}+e\nu, \pi, \rho$}
\begin{frame}
  \frametitle{$B_c\to \chi_{cJ}+\R$}
\begin{center}
  \includegraphics[width=0.5\columnwidth]{diags_BcCCW}
\end{center}
$$\M\left(B_c \to \chi_{cJ} + \R\right) = H^\mu \epsilon^{(\R)}_\mu$$
Form-factors of the $B_c\to \chi_{cJ}W$ transitions were considered in
\begin{itemize}
\item D. Ebert, R. N. Faustov, V. O. Galkin, PRD 82 (2010) 034019 
\item Zhi-hui Wang et al,  	J. Phys. G,  39 (2012) 015009
\item E. Hernandez et al, PRD      74 (2006) 074008
\item etc.
\end{itemize}
I will use these form factors, compare published branching fractions and consider some other decay modes
\end{frame}

\subsection{$\chi_{c0}$}
\begin{frame}
  \frametitle{$\chi_{c0}$, Form Factors}
  $$
  H_\mu = f_{+}\left(q^2\right) \left(p_1+p_2\right)_\mu + f_{-}\left(q^2\right) \left(p_1-p_2\right)_\mu 
  $$
  \includegraphics[width=0.9\textwidth]{figs/ff_chi_c0}

  $f_{-}(q^2)$ gives no contribution to most of the decays
  
  $f_{+}(q^2)$ are proportional to each other
\end{frame}

\newcommand{\Br}{\mathrm{Br}}

\begin{frame}
  \frametitle{$B_c \to \chi_{c0} \pi$}
  \begin{columns}
    \begin{column}{0.5\textwidth}
      \centering{[Ebert]:}
      % Bc -> chi_c0 + P with [Ebert]
      \begin{tabular}{lcr}
          Paper &:& $\Br  = 0.021\%$ \\
          this      &:& $\Br  = 0.016\%$ \\        
      \end{tabular}

    \end{column}
    \begin{column}{0.5\textwidth}
      \centering{[Wang]:}
      % Bc -> chi_c0 + P with [Wang]
      \begin{tabular}{lcr}
          Paper &:& $\Br  = 0.031\pm 0.004\%$ \\
          this      &:& $\Br  = 0.035\%$ \\
		  ratio   &:& $1.127\pm 0.145$ \\
      \end{tabular}

    \end{column}
  \end{columns}
  \vspace{2cm}
  Good agreement
\end{frame}

\newcommand{\VSpace}{}

\begin{frame}[t]
  \frametitle{$B_c \to \chi_{c0} e \nu$}
  \begin{columns}
    \begin{column}{0.5\textwidth}
      \centering{[Ebert]}
      % Bc -> chi_c0 + enu with [Ebert]
      \begin{tabular}{lcr}
          Paper &:& $\Br  = 0.087\%$ \\
          this      &:& $\Br  = 0.087\%$ \\        
      \end{tabular}

      \VSpace 
      \visible<2->{      \includegraphics[width=0.9\textwidth, height = 0.55\textheight]{chi_c0_enu_q2_Ebert}}
    \end{column}
    \begin{column}{0.5\textwidth}
      \only<3->{
        \centering{[Wang]}
        % Bc -> chi_c0 + enu with [Wang]
      \begin{tabular}{lcr}
          Paper &:& $\Br  = 0.13\pm 0.03\%$ \\
          this      &:& $\Br  = 0.138\%$ \\
		  ratio   &:& $1.058\pm 0.244$ \\
      \end{tabular}

      }
      \VSpace 
      \only<3,4>{\visible<4>{      \includegraphics[width=0.9\textwidth, height = 0.55\textheight]{chi_c0_enu_e2_Wang}}}
      \only<5>{      \includegraphics[width=0.9\textwidth, height = 0.55\textheight]{chi_c0_enu_q2_Wang}}
    \end{column}
  \end{columns}
\end{frame}

\begin{frame}
  \frametitle{$B_c \to \chi_{c0} \rho$}
  $$
  \Br\left[B_c \to \chi_{c0} \rho\right] =
  6\pi^2 f_\rho^2 \left.
    \frac{d\Br\left(B_c\to\chi_{c0}+e\nu\right)}{dq^2}
  \right|_{q^2=m_\rho^2}
  $$
  \begin{columns}
    \begin{column}{0.5\textwidth}
      \centering{Ebert}
      \vspace{3mm}
      \centering{\includegraphics[width=0.9\textwidth]{chi_c0_enu_rho_Ebert}}
      % Bc -> chi_c0 + V with [Ebert]
      \begin{tabular}{lcr}
          Paper &:& $\Br  = 0.058\%$ \\
          this      &:& $\Br  = 0.032\%$ \\        
      \end{tabular}

    \end{column}
  \begin{column}{0.5\textwidth}
      \centering{Wang}\\
      \vspace{3mm}
      \centering{\includegraphics[width=0.9\textwidth]{chi_c0_enu_rho_Wang}}
      % Bc -> chi_c0 + V with [Wang]
      % Bc -> chi_c0 + V with [Wang]
      \begin{tabular}{lcr}
          Paper &:& $\Br  = 0.076\pm 0.009\%$ \\
          this      &:& $\Br  = 0.07\%$ \\
		  ratio   &:& $0.917\pm 0.109$ \\
      \end{tabular}

       \end{column}
\end{columns}
\end{frame}

\subsection{$\chi_{c1}$}
\begin{frame}
  \frametitle{$\chi_{c1}$, Form Factors}
  \begin{align*}
    H_\mu =& \frac{2i h_A(q^2)}{M_1+M_2}e_{\mu\nu\rho\sigma}\epsilon^\nu p_1^\rho p_2^\sigma
             + (M_1+M_2) h_{V_1}(q^2)\epsilon_\mu + \\
     & [h_{V_2}(q^2)p_{1\mu} + h_{V_3}(q^2)p_{2\mu}]\frac{\epsilon q}{M_1}
  \end{align*}
  \includegraphics[width=0.8\textwidth]{figs/ff_chi_c1}
\end{frame}

\begin{frame}
  \frametitle{$B_c \to \chi_{c1} \pi$}
  \begin{columns}
    \begin{column}{0.5\textwidth}
      \centering{[Ebert]:}
      % Bc -> chi_c1 + P with [Ebert]
      \begin{tabular}{lcr}
          Paper &:& $\Br  = 0.02\%$ \\
          this      &:& $\Br  = 0.001\%$ \\
		  ratio   &:& $0.031$ \\
      \end{tabular}

    \end{column}
    \begin{column}{0.5\textwidth}
      \centering{[Wang]:}
      % Bc -> chi_c1 + P with [Wang]
      \begin{tabular}{lcr}
          Paper &:& $\Br  = 0.002\pm 0.\%$ \\
          this      &:& $\Br  = 0.003\%$ \\
		  ratio   &:& $1.311\pm 0.125$ \\
      \end{tabular}

    \end{column}
  \end{columns}
\end{frame}


\begin{frame}[t]
  \frametitle{$B_c \to \chi_{c1} e \nu$}
  \begin{columns}
    \begin{column}{0.5\textwidth}
      \centering{[Ebert]}
      % Bc -> chi_c1 + enu with [Ebert]
      \begin{tabular}{lcr}
          Paper &:& $\Br  = 0.082\%$ \\
          this      &:& $\Br  = 0.107\%$ \\
		  ratio   &:& $1.302$ \\
      \end{tabular}

      \VSpace 
      \visible<2->{      \includegraphics[width=0.9\textwidth, height = 0.55\textheight]{chi_c1_enu_q2_Ebert}}
    \end{column}
    \begin{column}{0.5\textwidth}
      \only<3->{
        \centering{[Wang]}
      % Bc -> chi_c1 + enu with [Wang]
      \begin{tabular}{lcr}
          Paper &:& $\Br  = 0.11\pm 0.03\%$ \\
          this      &:& $\Br  = 0.115\%$ \\
		  ratio   &:& $1.042\pm 0.284$ \\
      \end{tabular}

      }
      \VSpace 
      \only<3,4>{\visible<4>{      \includegraphics[width=0.9\textwidth, height = 0.55\textheight]{chi_c1_enu_e2_Wang}}}
      \only<5>{      \includegraphics[width=0.9\textwidth, height = 0.55\textheight]{chi_c1_enu_q2_Wang}}
    \end{column}
  \end{columns}
\end{frame}


\begin{frame}
  \frametitle{$B_c \to \chi_{c1} \rho$}
  $$
  \Br\left[B_c \to \chi_{c1} \rho\right] =
  6\pi^2 f_\rho^2 \left.
    \frac{d\Br\left(B_c\to\chi_{c1}+e\nu\right)}{dq^2}
  \right|_{q^2=m_\rho^2}
  $$
  \begin{columns}
    \begin{column}{0.5\textwidth}
      \centering{Ebert}
      \vspace{3mm}
      \centering{\includegraphics[width=0.9\textwidth]{chi_c1_enu_rho_Ebert}}
      % Bc -> chi_c1 + V with [Ebert]
      \begin{tabular}{lcr}
          Paper &:& $\Br  = 0.015\%$ \\
          this      &:& $\Br  = 0.013\%$ \\
		  ratio   &:& $0.845$ \\
      \end{tabular}

    \end{column}
  \begin{column}{0.5\textwidth}
      \centering{Wang}\\
      \vspace{3mm}
      \centering{\includegraphics[width=0.9\textwidth]{chi_c1_enu_rho_Wang}}
      % Bc -> chi_c1 + V with [Wang]
      \begin{tabular}{lcr}
          Paper &:& $\Br  = 0.023\pm 0.002\%$ \\
          this      &:& $\Br  = 0.018\pm 0.005\%$ \\        
      \end{tabular}

    \end{column}
\end{columns}
\end{frame}

\subsection{$\chi_{c2}$}
\begin{frame}
  \frametitle{$\chi_{c2}$, Form Factors}
  \begin{align*}
  H_\mu &=
\frac{2it_V(q^2)}{M_1+M_2} \epsilon^{\mu\nu\rho\sigma}\epsilon^*_{\nu\alpha}
          \frac{p_1^\alpha}{M_1}  p_1 p_1  
     +  (M_1+M_2)t_{A_1}(q^2)\epsilon^{*\mu\alpha}\frac{p_{1\alpha}}{M_1} +\\
&  [t_{A_2}(q^2)p_1^\mu+t_{A_3}(q^2)p_2^\mu]\epsilon^*_{\alpha\beta}
\frac{p_1^\alpha p_1^\beta}{M_1^2} , 
  \end{align*}

\begin{center}
  \includegraphics[width=0.7\textwidth]{figs/ff_chi_c2}
\end{center}
\end{frame}

\begin{frame}
  \frametitle{$B_c \to \chi_{c2} \pi$}
  \begin{columns}
    \begin{column}{0.5\textwidth}
      \centering{[Ebert]:}
      % Bc -> chi_c2 + P with [Ebert]
      \begin{tabular}{lcr}
          Paper &:& $\Br  = 0.038\%$ \\
          this      &:& $\Br  = 0.04\%$ \\
		  ratio   &:& $1.052$ \\
      \end{tabular}

    \end{column}
    \begin{column}{0.5\textwidth}
      \centering{[Wang]:}
      % Bc -> chi_c2 + P with [Wang]
      \begin{tabular}{lcr}
          Paper &:& $\Br  = 0.021\pm 0.005\%$ \\
          this      &:& $\Br  = 0.018\pm 0.006\%$ \\        
      \end{tabular}

    \end{column}
  \end{columns}
\end{frame}


\begin{frame}[t]
  \frametitle{$B_c \to \chi_{c2} e \nu$}
  \begin{columns}
    \begin{column}{0.5\textwidth}
      \centering{[Ebert]}
      % Bc -> chi_c2 + enu with [Ebert]
      \begin{tabular}{lcr}
          Paper &:& $\Br  = 0.16\%$ \\
          this      &:& $\Br  = 0.209\%$ \\
		  ratio   &:& $1.303$ \\
      \end{tabular}

      \VSpace 
      \visible<2->{      \includegraphics[width=0.9\textwidth, height = 0.55\textheight]{chi_c2_enu_q2_Ebert}}
    \end{column}
    \begin{column}{0.5\textwidth}
      \only<3->{
        \centering{[Wang]}
      % Bc -> chi_c2 + enu with [Wang]
      \begin{tabular}{lcr}
          Paper &:& $\Br  = 0.1\pm 0.03\%$ \\
          this      &:& $\Br  = 0.097\%$ \\
		  ratio   &:& $0.971\pm 0.291$ \\
      \end{tabular}

      }
      \VSpace 
      \only<3,4>{\visible<4>{      \includegraphics[width=0.9\textwidth, height = 0.55\textheight]{chi_c2_enu_e2_Wang}}}
      \only<5>{      \includegraphics[width=0.9\textwidth, height = 0.55\textheight]{chi_c2_enu_q2_Wang}}
    \end{column}
  \end{columns}
\end{frame}


\begin{frame}
  \frametitle{$B_c \to \chi_{c2} \rho$}
  $$
  \Br\left[B_c \to \chi_{c2} \rho\right] =
  6\pi^2 f_\rho^2 \left.
    \frac{d\Br\left(B_c\to\chi_{c2}+e\nu\right)}{dq^2}
  \right|_{q^2=m_\rho^2}
  $$
  \begin{columns}
    \begin{column}{0.5\textwidth}
      \centering{Ebert}
      \vspace{3mm}
      \centering{\includegraphics[width=0.9\textwidth]{chi_c2_enu_rho_Ebert}}
      % Bc -> chi_c2 + V with [Ebert]
      \begin{tabular}{lcr}
          Paper &:& $\Br  = 0.11\%$ \\
          this      &:& $\Br  = 0.062\%$ \\        
      \end{tabular}

    \end{column}
  \begin{column}{0.5\textwidth}
      \centering{Wang}\\
      \vspace{3mm}
      \centering{\includegraphics[width=0.9\textwidth]{chi_c2_enu_rho_Wang}}
      % Bc -> chi_c2 + V with [Wang]
      \begin{tabular}{lcr}
          Paper &:& $\Br  = 0.056\pm 0.011\%$ \\
          this      &:& $\Br  = 0.154\%$ \\
		  ratio   &:& $2.754\pm 0.541$ \\
      \end{tabular}

    \end{column}
\end{columns}
\end{frame}

\subsection{Results}
\begin{frame}
  \frametitle{Results}
    Original branchng fractions:
  {\tiny$$
\begin{array}{r|rr|rr|rr}
  &\multicolumn{2}{c}{\chi_{c0}} & \multicolumn{2}{c}{\chi_{c1}} & \multicolumn{2}{c}{\chi_{c2}} \\
   & \text{[Ebert]} & \text{[Wang]} & \text{[Ebert]} & \text{[Wang]} & \text{[Ebert]} & \text{[Wang]} \\
\hline
 e\nu & 0.087 & 0.13\pm 0.03 & 0.082 & 0.11\pm 0.03 & 0.16 & 0.1\pm 0.03 \\
 \pi & 0.021 & 0.031\pm 0.004 & 0.02 & 0.0021\pm 0.0002 & 0.038 & 0.021\pm 0.005 \\
 \rho & 0.058 & 0.076\pm 0.009 & 0.015 & 0.023\pm 0.002 & 0.11 & 0.056\pm 0.011 \\
\end{array}
$$}
  
  Predicitons for branching fractions of the considered decays
  {\tiny$$
\begin{array}{r|rr|rr|rr}
  &\multicolumn{2}{c}{\chi_{c0}} & \multicolumn{2}{c}{\chi_{c1}} & \multicolumn{2}{c}{\chi_{c2}} \\
   & \text{[Ebert]} & \text{[Wang]} & \text{[Ebert]} & \text{[Wang]} & \text{[Ebert]} & \text{[Wang]} \\
\hline
 e\nu & 0.0889 & 0.138 & 0.0822 & 0.115 & 0.16 & 0.0971 \\
 \pi & 0.0217 & 0.0349 & 0.000644 & 0.00281 & 0.041 & 0.0238 \\
 \rho & 0.0547 & 0.0901 & 0.0127 & 0.0269 & 0.105 & 0.0656 \\
 % 2\pi & 0.0568 & 0.0935 & 0.0154 & 0.0311 & 0.109 & 0.0684 \\
 % 3\pi & 0.0213 & 0.0345 & 0.0156 & 0.0249 & 0.041 & 0.0262 \\
 % 5\pi & 0.0000379 & 0.0000564 & 0.0000486 & 0.0000634 & 0.0000671 & 0.0000391 \\
\end{array}
$$}

  Ratios (this/paper)
  {\tiny$$
\begin{array}{r|rr|rr|rr}
  &\multicolumn{2}{c}{\chi_{c0}} & \multicolumn{2}{c}{\chi_{c1}} & \multicolumn{2}{c}{\chi_{c2}} \\
   & \text{[Ebert]} & \text{[Wang]} & \text{[Ebert]} & \text{[Wang]} & \text{[Ebert]} & \text{[Wang]} \\
\hline
 e\nu & 1.02 & 1.06\pm 0.244 & 1. & 1.04\pm 0.284 & 1. & 0.971\pm 0.291 \\
 \pi & 1.03 & 1.13\pm 0.145 & 0.0322 & 1.34\pm 0.128 & 1.08 & 1.14\pm 0.27 \\
 \rho & 0.943 & 1.19\pm 0.14 & 0.845 & 1.17\pm 0.102 & 0.955 & 1.17\pm 0.23 \\
\end{array}
$$}
  

\end{frame}


\section{$B_c\to \chi_{cJ}+n\pi$}
\begin{frame}
  \frametitle{$W\to \R$}
  $W\to \R$ amplitude can be calculated using
  \begin{itemize}
  \item ChPt
  \item resonance approximation
  \item Phenomenology
  \item etc
  \end{itemize}
  Spectral functions are defined as
  $$
  \int\frac{d\Phi(q \to k_1\dots k_n)}{2pi} \epsilon_\mu \epsilon^*_\nu = (q_\mu q_\nu - q^2 g_{\mu\nu}) \rho_T(q^2) + q_\mu q_\nu \rho_L(q^2)
  $$

  Checked comaring with
  \begin{itemize}
  \item $e^+ e^- \to \R$
  \item $\tau \to \nu_\tau \R$
  \item etc
  \end{itemize}
\end{frame}

\subsection{$2\pi$}
\begin{frame}
  \frametitle{$W \to  2\pi$}
  \begin{columns}
    \begin{column}{0.5\textwidth}
      % diag grabbed from arXiv:1104.0808
      \centering{\includegraphics[width=0.9\textwidth]{diag_W_2pi}}
    \end{column}
    \begin{column}{0.5\textwidth}
  \includegraphics[width=0.9\textwidth]{figs/rhoT_2pi_q2}
    \end{column}
  \end{columns}
\end{frame}

\begin{frame}
  \frametitle{$B_c \to \chi_{cJ} + 2\pi$}
  \centering{\includegraphics[width=0.6\textwidth]{2pi_q2}}

  $B_c \to \chi_{cJ} + 2\pi$ decay branching fractions (in $10^{-2}$) 

  $$
\begin{array}{c|cc}
 \text{} & \text{[Ebert]} & \text{[Wang]} \\
\hline
 \chi_{\text{c0}} & 0.057 & 0.094 \\
 \chi_{\text{c1}} & 0.015 & 0.031 \\
 \chi_{\text{c2}} & 0.11 & 0.068 \\
\end{array}$$

\end{frame}

\subsection{$3\pi$}
\begin{frame}
  \frametitle{$W \to  3\pi$}
  \begin{columns}
    \begin{column}{0.5\textwidth}
      % diag grabbed from arXiv:1104.0808
      \centering{\includegraphics[width=0.9\textwidth]{diag_W_3pi}}
    \end{column}
    \begin{column}{0.5\textwidth}
  \includegraphics[width=0.9\textwidth]{figs/rhoT_3pi_q2}
    \end{column}
  \end{columns}
\end{frame}

\begin{frame}
  \frametitle{$B_c \to \chi_{cJ} + 3\pi$}
  \centering{\includegraphics[width=0.6\textwidth]{3pi_q2}}

  $B_c \to \chi_{cJ} + 3\pi$ decay branching fractions (in $10^{-2}$) 

  $$
\begin{array}{c|cc}
 \text{} & \text{[Ebert]} & \text{[Wang]} \\
\hline
 \chi_{\text{c0}} & 0.021 & 0.034 \\
 \chi_{\text{c1}} & 0.016 & 0.025 \\
 \chi_{\text{c2}} & 0.041 & 0.026 \\
\end{array}$$

\end{frame}


\subsection{$5\pi$}
\begin{frame}
  \frametitle{$W \to  5\pi$}
  \begin{columns}
    \begin{column}{0.5\textwidth}
      % diag grabbed from arXiv:1104.0808
      \centering{\includegraphics[width=0.9\textwidth]{diag_W_5pi}}
    \end{column}
    \begin{column}{0.5\textwidth}
  \includegraphics[width=0.9\textwidth]{figs/rhoT_5pi_q2}
    \end{column}
  \end{columns}
\end{frame}


\begin{frame}
  \frametitle{$B_c \to \chi_{cJ} + 5\pi$}
  \centering{\includegraphics[width=0.6\textwidth]{5pi_q2}}
  
$B_c \to \chi_{cJ} + 5\pi$ decay branching fractions (in $10^{-6}$) 

$$
\begin{array}{c|cc}
 \text{} & \text{[Ebert]} & \text{[Wang]} \\
\hline
 \chi_{\text{c0}} & 0.38 & 0.56 \\
 \chi_{\text{c1}} & 0.49 & 0.63 \\
 \chi_{\text{c2}} & 0.67 & 0.39 \\
\end{array}$$

\end{frame}

\section{Results and Conclusion}
\begin{frame}
  \frametitle{Conclusion}
  \begin{itemize}
  \item Considering $B_c \to \chi_{cJ}+\R$ decays
  \item Factorization model, spectral function approach
  \item Form-factor sets from
    \begin{itemize}
    \item Ebert
    \item Wang
    \end{itemize}
  \item Resonable agreement with papers' results, with one exception
    \begin{itemize}
    \item [Ebert]: $B_c\to\chi_{c1}\pi$
    \end{itemize}
  \item New results for $R=2\pi, 3\pi, 5\pi$
  \item New decays added to EvtGen model (BC\_VHAD)
  \end{itemize}
  \vspace{1.5cm}
  \centering{Thank you for your attention}
\end{frame}


\end{document}
